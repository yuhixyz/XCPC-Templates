%!TEX program=xelatex
\documentclass[10pt,a4paper]{article}
%\usepackage{zh_CN-Adobefonts_external}
%\usepackage{xeCJK}
\usepackage{ctex}
\usepackage{amsmath, amsthm}
\usepackage{listings,xcolor}
\usepackage{geometry} % 设置页边距
\usepackage{fontspec}
\usepackage{graphicx}
\usepackage[colorlinks]{hyperref}
\usepackage{setspace}
\usepackage{fancyhdr} % 自定义页眉页脚
\usepackage[toc]{multitoc} % 多栏目录,默认两栏

\hypersetup{linkcolor=black} % 设置目录为黑色

\setsansfont{Consolas} % 设置英文字体
\setmonofont[Mapping={}]{Courier New} % 英文引号之类的正常显示,相当于设置英文字体
\setlength{\parindent}{0pt} % 取消文章所有的首行缩进

\linespread{1.2}

\title{Templates For ICPC}
\author{Yunhai Bian @ NJUPT}
\definecolor{dkgreen}{rgb}{0,0.6,0}
\definecolor{gray}{rgb}{0.5,0.5,0.5}
\definecolor{mauve}{rgb}{0.58,0,0.82}

\pagestyle{fancy}

\lhead{\CJKfamily{kai} Nanjing University of Posts and Telecommunications} %以下分别为左中右的页眉和页脚
\chead{}

\rhead{\CJKfamily{kai} 第 \thepage 页}
\lfoot{} 
\cfoot{\thepage}
\rfoot{}
\renewcommand{\headrulewidth}{0.4pt} 
\renewcommand{\footrulewidth}{0.4pt}
%\geometry{left=2.5cm,right=3cm,top=2.5cm,bottom=2.5cm} % 页边距
\geometry{left=3.18cm,right=3.18cm,top=2.54cm,bottom=2.54cm}
\setlength{\columnsep}{30pt}

\makeatletter

\makeatother


\lstset{
    language    = c++,
    numbers     = left,
    numberstyle={                               % 设置行号格式
        \small
        \color{black}
        \fontspec{Courier New}
    },
	commentstyle = \color{black}, %代码注释的颜色
	keywordstyle={                              % 设置关键字格式
        \color{black}
        \fontspec{Courier New Bold}
        \bfseries
    },
	stringstyle={                               % 设置字符串格式
        \color{black}
        \fontspec{Courier New}
    },
	basicstyle={                                % 设置代码格式
        \fontspec{Courier New}
        \small\ttfamily
    },
	emphstyle=\color[RGB]{112,64,160},          % 设置强调字格式
    breaklines=true,                            % 设置自动换行
    tabsize     = 4,
    frame       = single,%主题
    columns     = fullflexible,
    rulesepcolor = \color{red!20!green!20!blue!20}, %设置边框的颜色
    showstringspaces = false, %不显示代码字符串中间的空格标记
	escapeinside={\%*}{*)},
}

\begin{document}
\title{NJUPT XCPC Templates}
\author {Yunhai Bian}
\maketitle
\tableofcontents
\newpage
\section{数据结构}
\subsection{树状数组}
\lstinputlisting{数据结构/树状数组.cpp}
\subsection{线段树}
\subsubsection{HDU1540}
\begin{spacing}{1.5}
题意:有  个点连成一条线,编号从左至右为 ,有三种操作:① 摧毁一个点 ② 查询某个点能到的所有点数(包括自己) ③ 重建上一次被摧毁的点。

分析:用一个栈 stk 来存放被摧毁的点,摧毁点 x 就 stk[++top] = x,重建上一个点就只需要取出栈顶 x = stk[top--]。线段树每个节点维护区间左侧连续最大长度(点数) lmax 以及右侧最大连续长度 rmax。摧毁一个点就是在线段树中找到该点并将其 lmax=rmax=0,重建就是 lmax=rmax=1,然后 pushup 上去。难点在于②号查询操作,如果点 x 在当前结点的左孩子,分两种情况来看,如果点 x 被左孩子的右侧最大连续区间包含了,那么 x 能到达的所有点数就是左孩子的 rmax + 右孩子的 lmax,否则递归直接递归左孩子即可。剩余情况类似。
\end{spacing}
\lstinputlisting{数据结构/线段树/HDU1540.cpp}
\subsubsection{HDU4578}
\begin{spacing}{1.5}
题意:线段树区间加,区间乘,区间置数,区间和,平方和,立方和。

分析:需要维护,置数标记 same,乘法标记 mul,加法标记 add,区间和标记 s[0\~2] 分别表示和,平方和,立方和。

首先确定前三个标记维护优先级,same > mul > add,然后就是三个和的维护需要推导一下。

1. 区间置数,三个和很好维护不说了。
2. 区间乘 $k$,三个和分别乘以 $k,k^2,k^3$
3. 区间加 $a$,初始有 $s[0]=\sum{x}$, $s[1]=\sum{x^2}$, $s[2]=\sum{x^3}$,区间长度为 $len$.
    \begin {equation} 
    \sum{(x+a)}=\sum{x}+\sum{a}=s[0]+len*a
    \end {equation}
    \begin {equation} 
    \sum{(x+a)^2}=\sum{x^2}+2a\sum{x}+\sum{a^2}=s[1]+2a*s[0]+len*a^2 \\
    \end {equation}
    \begin {equation} 
    \sum{(x+a)^3}=\sum{x^3}+3a\sum{x^2}+3a^2\sum{x}+\sum{a^3}=s[2]+3a*s[1]+3a^2*s[0]+len*a^3 
    \end {equation}

    注意维护和的时应该倒序维护(立方和,平方和,和),防止要用的值被先更新了。
\end{spacing}
\lstinputlisting{数据结构/线段树/HDU4578.cpp}
\subsubsection{HDU4553}
\begin{spacing}{1.5}
题意:有一个长度为 n 的时间轴,有两种操作:1. DS QT 表示屌丝申请第一段长度为 QT 的空闲时间,能申请到就输出起始时间。2. NS QT 表示女神申请第一段长度为 QT 的空闲时间,如果能申请到输出起始时间,如果找不到,可以无视屌丝已经申请的时间,再找到一个第一个连续空闲时间大于等于 QT 的起始位置。3. STUDY!! L R 表示清空这段时间的所有申请用于学习,由于三分钟热度,之后再有人申请到 STUDY 的时间还是会分配出去。

分析:线段树维护两个时间轴的信息,分别表示屌丝时间轴的分配情况,还有女神时间轴的分配情况。详见代码,下标 0 表示屌丝,下标 1 表示女神。same 为区间相同的值的标记,lmax, rmax, tmax 分别表示区间左侧最长连续空闲时间,右侧最长连续空闲时间,区间内最长连续空闲时间,用 1 表示空闲。然后根据题目要求操作即可。
\end{spacing}
\lstinputlisting{数据结构/线段树/HDU4553.cpp}
\subsubsection{HDU1542}
\begin{spacing}{1.5}
题意:线段树扫描线求矩形面积并。

分析:注意线段树的每一个叶子结点表示的不是单个点,而是一个区间,其中的标记含义如注释。
\end{spacing}
\lstinputlisting{数据结构/线段树/HDU1542.cpp}
\subsubsection{HDU1255}
\begin{spacing}{1.5}
题意:线段树扫描线求至少被覆盖 2 次的矩形面积并。

分析:与上一个题目类似,这里需要分别维护 len1, len2,其中 len1 含义与上一个题的 len 一样,len2 表示线段树区间内被覆盖至少 2 次的实际区间的合并的长度。只需要求改 pushup 函数,更新 len2 时候需要分情况讨论,如果区间被完全覆盖了至少 2 次,len2 就是区间长度;否则,如果当前是叶子结点,那么此时最多会被完全覆盖 1 次,对 len2 没有贡献;否则,如果不是叶子结点并且恰好被覆盖 1 次,那么想要求该区间内至少被覆盖 2 次的长度,就需要计算当前结点的左右子结点中被覆盖至少 1 次的长度,如果不是叶子结点并且没有被完全覆盖过,直接用子结点的 len2 之和来更新当前结点的 len2 即可。有点绕,但是并不难理解。
\end{spacing}
\lstinputlisting{数据结构/线段树/HDU1255.cpp}
\subsection{ST表}
\lstinputlisting{数据结构/RMQ.cpp}
\subsection{Splay}
\lstinputlisting{数据结构/Splay.cpp}
\section{图论}
\subsection{最短路}
\subsection{最小生成树}
\subsection{次小生成树}
\subsection{有向图的强连通分量}
\subsection{无向图的双连通分量}
\subsection{最近公共祖先}
\subsection{2-SAT}
\subsection{网络流}
\subsubsection{Dinic}
\begin{spacing}{1.5}
\input{图论/网络流/Dinic.tex}
\end{spacing}
\lstinputlisting{图论/网络流/Dinic.cpp}
\subsubsection{EK}
\lstinputlisting{图论/网络流/EK.cpp}
\subsection{二分图}
\section{其他}
\subsection{高精度}
\lstinputlisting{其他/高精度.cpp}
\subsection{莫队}
\end{document}