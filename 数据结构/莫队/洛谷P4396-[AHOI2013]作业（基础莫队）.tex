对于每个询问区间 [l, r] 需要求在该区间内值域在 [a, b] 上的数的个数以及不同的数的个数。

可以考虑用两个树状数组来维护当前区间中出现的数字的个数,和不同数字的个数,然后差分一下就得到某个值域中出现的次数了。但这样插入和查询都是 $O(\log n)$ 。加上莫队总复杂度就达到了 $O(n\sqrt{n}\log n)$,这是无法接受的。

考虑值域分块,然后维护每一块的和。插入就是 $O(1)$ 查询为 $O(\sqrt{n})$,不会影响总复杂度。