回滚莫队,一般用在当区间维护的答案只具有“可加性”或者只具有“可减性”时,这里只讨论,只具有“可加性”的情况。对于此类情况,回滚莫队能将删除操作 del 全部转化为插入操作 add 。

依次处理询问,我们对询问进行分段处理,把左端点处于同一块的询问放在一起处理。

对于这些左端点处于同一块的询问来说,它们的右端点递增,我们再细分为两种情况。

1. 左右端点在同一块内:直接暴力做就行了,l, r 指针移动是 O(n) 的。

2. 左右端点跨块,分为两部分:左端点所属块的部分,和右边的部分,初始化区间 r = R[belong[q[i].l]], l = r + 1,右端点向右一直 add,左端点向左 add,每次做完左端点需要归位并消除影响。

取块大小为 $B=\sqrt{n}$ 的话,总复杂度为 $O(n\sqrt{n}+m\sqrt{n})$ 。

