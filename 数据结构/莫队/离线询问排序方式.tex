一般莫队排序方式:以 belong[l] 为第一关键字, r 为第二关键字升序排序。

\begin{lstlisting}
struct Query {
    int id, l, r;
    bool operator < (const Query &W) const {
        if (belong[l] != belong[W.l]) return belong[l] < belong[W.l];
        return r < W.r;
    }
} q[M];
// 奇偶优化
struct Query {
	int id, l, r;
	bool operator < (const Query &W) const {
        if (belong[l] != belong[W.l]) return belong[l] < belong[W.l];
        return belong[l] & 1 ? r < W.r : r > W.r;  
	}
} q[N];
\end{lstlisting}
    

带修莫队排序方式:以 belong[l] 为第一关键字, belong[r] 为第二关键字, ts 为第三关键字升序排序。

\begin{lstlisting}
struct Query {
    int id, l, r, ts; // id表示当前询问的编号,ts表示当前询问处于第ts次操作后,第ts+1操作前
    bool operator < (const Query &W) const {
        if (belong[l] != belong[W.l]) return belong[l] < belong[W.l];
        if (belong[r] != belong[W.r]) return belong[r] < belong[W.r];
        return ts < W.ts;
    }
} q[N];
\end{lstlisting}
