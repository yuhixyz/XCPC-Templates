题意:有一个长度为 n 的时间轴,有两种操作:① DS QT 表示屌丝申请第一段长度为 QT 的空闲时间,能申请到就输出起始时间。② NS QT 表示女神申请第一段长度为 QT 的空闲时间,如果能申请到输出起始时间,如果找不到,可以无视屌丝已经申请的时间,再找到一个第一个连续空闲时间大于等于 QT 的起始位置。③ STUDY!! L R 表示清空这段时间的所有申请用于学习,由于三分钟热度,之后再有人申请到 STUDY 的时间还是会分配出去。

分析:线段树维护两个时间轴的信息,分别表示屌丝时间轴的分配情况,还有女神时间轴的分配情况。详见代码,下标 0 表示屌丝,下标 1 表示女神。same 为区间相同的值的标记,lmax, rmax, tmax 分别表示区间左侧最长连续空闲时间,右侧最长连续空闲时间,区间内最长连续空闲时间,用 1 表示空闲。然后根据题目要求操作即可,代码中相关注释应该比较清楚。