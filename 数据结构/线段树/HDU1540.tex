题意:有 n 个点连成一条线,编号从左至右为 1\~{}n,有三种操作:1. 摧毁一个点 2. 查询某个点能到的所有点数(包括自己) 3. 重建上一次被摧毁的点。

分析:用一个栈 stk 来存放被摧毁的点,摧毁点 x 就 stk[++top] = x,重建上一个点就只需要取出栈顶 x = stk[top--]。线段树每个节点维护区间左侧连续最大长度(点数) lmax 以及右侧最大连续长度 rmax。摧毁一个点就是在线段树中找到该点并将其 lmax=rmax=0,重建就是 lmax=rmax=1,然后 pushup 上去。难点在于②号查询操作,如果点 x 在当前结点的左孩子,分两种情况来看,如果点 x 被左孩子的右侧最大连续区间包含了,那么 x 能到达的所有点数就是左孩子的 rmax + 右孩子的 lmax,否则递归直接递归左孩子即可。剩余情况类似。